\chapter{Verfahrensbeschreibung}\label{ch:verfahrensbeschreibung}


\section{Gesamtsystem}\label{sec:gesamtsystem}
Das System arbeitet nach dem \textbf{EVA}-Prinzip.
Die einzelnen Komponenten laufen jeweils in Threads.
Die \textbf{EVA}-Segmente werden von einem Controller koordiniert, welcher gleichzeitig auch der Einstiegspunkt des Programms ist.
Zu Beginn des Programms nimmt der Controller per Argument einen Ordnerpfad entgegen.
Anschließend startet er die einzelnen Komponenten als Threads.

\subsection{Eingabe}\label{subsec:eingabe}
Sobald er gestartet wurde, liest der Eingabethread permanent Dateien aus dem übergebenen Ordner ein.
Er selbst führt eine Queue, welche alle bereits eingelesenen Dateien enthält.
Alle 0.05s wird von einem weiteren Thread, dessen einzige Aufgabe es ist, eine Methode aufzurufen ein, nach außen sichtbares, Model ersetzt.
Dadurch wird der gewünschte Effekt simuliert, dass der Detektor mit 20Hz immer eine andere Messreihe zur Verfügung stellt.
Der Controller schaut permanent, ob es ein sichtbares Objekt gibt und gibt dieses an die Verarbeitung weiter, falls sich das sichtbare Objekt geändert hat.

\subsection{Verarbeitung}\label{subsec:verarbeitung}
Die Verarbeitungskomponente führt ebenfalls eine Queue, welche von außen befüllt werden kann.
Gibt es ein neues Objekt zum Verarbeiten in der Queue, schaut sie, ob zu diesem Objekt bereits eine Ausgabe existiert.
Ist das nicht der Fall, wird der~\nameref{sec:mathematische-methoden}-Prozess gestartet.
Nach erfolgreicher Berechnung wird das Objekt an die Ausgabeinstanz weitergegeben.

\subsection{Ausgabe}\label{subsec:ausgabe}
Die Queue der Ausgabeinstanz wird von der Verarbeitung befüllt.
Gibt es ein neues Objekt und wurde dieses bisher noch nicht als Datei veröffentlicht, startet sie den Ausgabeprozess.
Hierbei wird das berechnete Modell in einer vorgegebenen Struktur in eine~.txt-Datei geschrieben.


\section{Strukturen}\label{subsec:strukturen}
Sowohl die Eingabe als auch die Ausgabe implementieren jeweils ein Interface, welches Runnable erweitert.
Durch das Interface sind implementierende Klassen gezwungen sowohl die Funktion einer Ein-/Ausgabe als auch die eines Runnables zur Verfügung zu stellen.
Runnables sind Objekte, die in einem Thread gestartet werden können.
Dies ermöglicht eine einfache Austauschbarkeit der Komponenten, welche im Controller initialisiert werden.

\subsection{Datenstruktur}\label{subsec:datenstruktur}
Eingelesene Dateien werden in einem Datensatz in einer Liste von Datenpunkten vom Typ Integer gespeichert.
Ein Datenpunkt ist ein generisches Objekt, welches drei Attribute hält.
Diese 3 Attribute sind vom selben Typ, welcher beim Erstellen eines Objekts festgelegt wird.
Im Datenpunkt werden die Position des Spiegels und die Intensität gespeichert.
Nach der Verarbeitung eines Datensatzes enthält dieser eine weitere Liste an Datenpunkten vom Typ Double.
Diese Datenpunkte haben als drittes Attribut zusätzlich den Wert der oberen Einhüllenden befüllt.
Im Datensatz werden außerdem Dateiname, Pulsbreite und das Maximum der Intensität gespeichert.\\
Siehe auch:
\protect\newpage
\section{Mathematische Methoden}\label{sec:mathematische-methoden}
