\chapter{Zusammenfassung und Ausblick}\label{ch:zusammenfassung-und-ausblick}


\section{Zusammenfassung}\label{sec:zusammenfassung}
Es wurde ein System angefertigt, welches den algorithmischen Kern zur Bestimmung der Autokorrelationsfunktion, für die Auswertung eines Signals, umsetzt.
Die einzelnen Komponenten, bestehend aus einer Eingabesimulation, einer Verarbeitung und der Dateiausgabe, wurden nebenläufig realisiert, wodurch diese unabhängig voneinander arbeiten können.
Außerdem wurde darauf geachtet, dass die Komponenten einfach auszutauschen sind und gegebenenfalls skaliert werden können.
Es wurde eine sinnvolle, zweite Ausgabetechnik bereits implementiert, welche die Idee eines eventbasierten Systems skizziert.\\

Dieses Dokument beschreibt zunächst die Anforderungen an das Programm und präzisiert die Aufgabenstellung.
Danach folgen das Verfahren des Gesamtsystems sowie die mathematischen Modelle.
Weitergehend bietet es eine Vielzahl an Diagrammen, um dem Interessenten einzelne Abläufe im Programm zu veranschaulichen.


\section{Ausblick}\label{sec:ausblick}
Für die Zukunft, könnte das Programm stärker eventbasiert arbeiten.
Zudem kann, wie auch bei der Ausgabe bereits skizziert, die Verarbeitung skaliert werden.
Insbesondere wenn größere Dateien verarbeitet werden sollen, wird die Berechnungszeit eines Datensatzes enorm ansteigen.
Dies wird dazu führen, dass eingelesene Daten wieder verworfen werden, da die Queue der Verarbeitungsinstanz voll ist.
Durch Anpassen des QUEUE\_LIMITs von aktuell 100 Datensätzen, kann dies zwar hinausgezögert werden, aber anstatt dauerhaft in einer while-Schleife zu überprüfen, ob ein Objekt zum Verarbeiten existiert, kann hier auch stattdessen pro Berechnung ein Thread initialisiert werden.
Es würde allerdings schon ausreichen mehrere Berechnungs-Threads gleichzeitig laufen zu haben um auf diese aufzuteilen.
Das Programm bedarf weitergehend einer stärkeren Testung.
Für Tests sind weitere simulierte Daten notwendig um besonders Grenzfälle provozieren zu können und das Verhalten des Programms dahingehend zu testen.

Außerdem ist die Ausgabe in eine~.txt-Datei für eine realistische Nutzung unwirksam.
Wenn man bedenkt, dass alle 0.05s eine Messreihe zur Verfügung gestellt wird, würden nach 10s Laufzeit bereits 200 Dateien existieren.
Hier wäre eine Ausgabe in Form eines Feeds beziehungsweise Streams denkbar, welche auf Abruf aktuelle Daten liefert.


