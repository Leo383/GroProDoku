\chapter{Aufgabenanalyse}\label{ch:aufgabenanalyse}


\section{Interpretation der Aufgabe}\label{sec:interpretation-der-aufgabe}
Gefordert ist ein Programm, welches einen Autokorrelator umsetzt.
Ein Autokorrelator ist ein Objekt, welches in diesem Fall ein Signal mit sich selbst zu einem späteren Zeitpunkt vergleicht.
Diese Untersuchung auf Selbstähnlichkeit führt zu einer Rauschminimierung des Signals.\\

Die Umsetzung erfolgt auf Basis von Simulationsdaten, welche per~.txt-Dateien bereitgestellt werden.
Jede Datei beinhaltet die Messdaten eines Signals.
Die einzelnen Zeilen beinhalten die Intensität des Signals und die dazugehörige Position des Spiegels, welcher im optischen Autokorrelator physisch bewegt wird.\\
Beginnt eine Zeile mit einem \enquote{\#} soll sie ignoriert werden.
Einzelne Werte einer Zeile werden mit einem Tabulator getrennt.\\
Dem Programm sollen mehrere Messdaten bereitgestellt werden können, da es im späteren Einsatz permanent Daten erhält.
Das Einlesen einer Datei soll als Thread geschehen.
Jede Datei wird der verarbeitenden Instanz für 0.05s zur Verfügung gestellt.
Die Verarbeitung einer Messreihe wird ebenfalls in einem Thread realisiert.
Sie erfolgt über eine mathematische Aufbereitung der Daten.
Dabei werden die Positionswerte des Spiegels in Picosekunden transformiert.
Die Intensität des Signals soll normiert werden, indem jeder Wert durch den höchsten Wert, also das Maximum aller Werte dividiert wird.\\
Darauf folgt eine Glättung der Daten, indem die transformierten Positionsdaten, also die Picosekunden, über den gleitenden Mittelwert eines Mittelungsfensters von $0.2\%$ der Gesamtdatenmenge berechnet, beziehungsweise verändert werden.\\
Für die Randwerte, um welche keine Mittelungsfenster erzeugt werden können, wird jeweils der letztmögliche Wert sooft dupliziert, wie es nötig ist, um das Fenster zu erzeugen.
Es wird also künstlich eine Konstante hinzugefügt.\\


Im nächsten Schritt der Aufbereitung soll die sogenannte \enquote{Obere Einhüllende} approximiert werden, indem von links annähernd bis zum globalen Maximum der Messreihe, jedem Positionswert $x_k$ der bisher höchste Messwert zugeordnet wird.
Das gleiche Verfahren wird dann von rechts wiederholt.\\
Mit dieser Einhüllenden kann anschließend die Pulsbreite berechnet werden.
Hierfür wird zuerst die Grundlinie berechnet, welche die Höhe im Mittel der ersten $0.01 * N$ Messwerte ist, wobei $N$ die Anzahl aller Elemente ist.
Die Pulsbreite ist dann die Distanz der Punkte, an dem die Einhüllende die halbe Höhe zwischen der Grundlinie und ihrem Maximum beträgt.\\

Im Anschluss an die Berechnung wird die Ausgabedatei erzeugt.
Das Erzeugen geschieht ebenfalls in einem Thread, welcher von der verarbeitenden Instanz angewiesen werden soll.
Die Ausgabedatei beginnt mit einer Kommentarzeile, welche die Pulsbreite und Indizes der äußeren Punkte der Pulsbreite kommasepariert enthält.
Darauf folge n in jeder Zeile die transformierten Positionen, die normierte Intensität des Signals und den Funktionswert der oberen Einhüllenden an der Stelle, wieder mit Tabulator getrennt.
Es können beliebig viele Kommentarzeilen auftreten, die erste Zeile muss allerdings die Kommentarzeile mit der Pulsbreite sein.
Der Name der Ausgabedatei soll mit \enquote{out} beginnen und im Anschluss den Dateinamen enthalten.
Ebenfalls ist die Ausgabedatei eine~.txt-Datei.\\
Das Programm ist beendet, wenn jede Eingabedatei verarbeitet und eine Ausgabedatei dazu erstellt wurde.

\begin{figure}[h]
    \centering
    \caption{Input-Restriktionen}
    \begin{itemize}[noitemsep]
        \item Es dürfen nur positive, ganzzahlige Werte
    \end{itemize}
    \label{fig:input-restrictions}
\end{figure}

Die Lösung des Problems wird mittels TODO realisiert. %TODO
TODO Beschreibe Algorithmus %TODO


\section{Fehlerarten}\label{sec:fehlerarten}
Die Eingabedatei kann verschiedene Integritätsbedingungen verletzen.
Das Programm muss diese Fehlerarten identifizieren und den Nutzer darüber informieren.

\subsection{Technische Fehler}\label{subsec:technische-fehler}
%TODO

\subsection{Syntaktische Fehler}\label{subsec:syntaktische-fehler}
Die Eingabedatei muss der Struktur aus~\nameref{fig:input-restrictions} entsprechen.
So kann zum Beispiel ein syntaktischer Fehler provoziert werden, indem TODO%TODO

\subsection{Semantische Fehler}\label{subsec:semantische-fehler}


\section{Fehlerbehandlung}\label{sec:fehlerbehandlung}

\subsection{Technische Fehler}\label{subsec:technische-fehler-behandlung}

\subsection{Syntaktische Fehler}\label{subsec:syntaktische-fehler-behandlung}

\subsection{Semantische Fehler}\label{subsec:semantische-fehler-behandlung}

\subsection{Sonderfälle}\label{subsec:sonderfaelle}
